\documentclass{report}
\usepackage[T1]{fontenc}
\usepackage[utf8]{inputenc}
\usepackage[francais]{babel}
\usepackage{graphicx}
\usepackage{url}

\title{Création du site web pour le BDE Agrosup Dijon}
\author{Dupin Nicolas - Grastilleur Julie}
\date{\today}
\pagestyle{headings}

\begin{document}

\section{Drupal vs Wordpress vs Joomla}

Pour faciliter la création du site web, nous avons choisi d'utiliser un CMS : Content Management System. Nous avons retenu les trois CMS les plus utilisées \cite{popularite1}, nous sommes sur de disposer de ressources, de plugins et d'aide suffisante pour correctement developper le site. Nous avons donc retenu Drupal, Wordpress et Joomla. Nous avons d'abord choisi d'éliminer Joomla de notre choix puisque celui-ci possède beaucoup moins de plugin et de theme que ses concurants \cite{comparatif1}. De plus il excèle dans son domaine pour des sites de e-commerce ou de réseaux sociaux, ce qui ne correspond pas à notre projet.


Le plus grand choix à été de choisir entre Drupal et Wordpress. 

\bibliographystyle{plain}
\bibliography{biblio}

\end{document}