\documentclass{report}
\usepackage[T1]{fontenc}
\usepackage[utf8]{inputenc}
\usepackage[francais]{babel}
\usepackage{graphicx}
\usepackage{url}

\title{Création du site web pour le BDE Agrosup Dijon}
\author{Dupin Nicolas - Grastilleur Julie}
\date{\today}
\pagestyle{headings}

\begin{document}

\section*{Drupal vs Wordpress vs Joomla}

Pour faciliter la création du site web, nous avons choisi d'utiliser un CMS : Content Management System. Nous avons retenu les trois CMS les plus utilisées \cite{popularite1}, nous sommes ainsi sur de disposer de ressources, de plugins et d'aides suffisantes pour correctement developper le site. Nous avons donc retenu Drupal, Wordpress et Joomla. Nous avons d'abord choisi d'éliminer Joomla de notre choix puisque celui-ci possède beaucoup moins de plugins et de thèmes que ses concurants \cite{comparatif1}. De plus il excèle dans son domaine pour des sites de e-commerce ou de réseaux sociaux, ce qui ne correspond pas à notre projet.\\


Le plus grand choix à été de choisir entre Drupal et Wordpress. Nous avons finalement choisi d'utiliser Wordpress pour plusieurs raisons. Il est récent, beaucoup utilisé, il possède une grande communauté, un nombre faramineux de plugins et il est facile à administrer \cite{pourquoiWP1, pourquoiWP2}. De l'autre côté, nous avons drupal qui peut effecter les mêmes choses mais requiert plus de compétences techniques. Au vu du cahier des charges et des fonctions "`basiques"' que nous voulons implenter, nous avons décidé d'utiliser Wordpress nous permettant d'éviter quelques complications inutiles.

\bibliographystyle{plain}
\bibliography{biblio}

\end{document}