\documentclass{report}
\usepackage[T1]{fontenc}
\usepackage[utf8]{inputenc}
\usepackage[francais]{babel}
\usepackage{graphicx}
\usepackage{url}

\title{Création du site web pour le BDE Agrosup Dijon}
\author{Dupin Nicolas - Grastilleur Julie}
\date{\today}
\pagestyle{headings}

\begin{document}

\section*{Drupal vs Wordpress vs Joomla}

Pour faciliter la création du site web, nous avons choisi d'utiliser un CMS : Content Management System. Nous avons retenu les trois CMS les plus utilisées \cite{popularite1}, nous sommes ainsi sur de disposer de ressources, de plugins et d'aides suffisantes pour correctement developper le site. Nous avons donc retenu Drupal, Wordpress et Joomla. Nous avons d'abord choisi d'éliminer Joomla de notre choix puisque celui-ci possède beaucoup moins de plugins et de thèmes que ses concurants \cite{comparatif1}. De plus il excèle dans son domaine pour des sites de e-commerce ou de réseaux sociaux, ce qui ne correspond pas à notre projet.\\


Le plus grand choix à été de choisir entre Drupal et Wordpress. Nous avons finalement choisi d'utiliser Wordpress pour plusieurs raisons. Il est récent, beaucoup utilisé, il possède une grande communauté, un nombre faramineux de plugins et il est facile à administrer \cite{pourquoiWP1, pourquoiWP2}. De l'autre côté, nous avons drupal qui peut effecter les mêmes choses mais requiert plus de compétences techniques. Au vu du cahier des charges et des fonctions "`basiques"' que nous voulons implenter, nous avons décidé d'utiliser Wordpress nous permettant d'éviter quelques complications inutiles.

\newpage
\section*{Gestion des photos}
La gestion des photos est l'un des éléments les plus délicats de ce projet. En effet, le BDE Agrosup poste un très grand nombre de photo sur son site. Il nous a donc fallu réfléchir à la meilleure solution possible. La première question a donc été de savoir où les photos allaient être hébergées. L'hébergeur actuel du site (qui sera conservé pour ce projet) proposant un espace de stockage illimité, nous en avons conlue que la meilleure solution était encore d'enregistrer les photos directement dans les dossiers du site, sans faire appel à un service d'hébergement tiers.\\

Une fois cette décision prise, nous avons réfléchi aux besoins des utilisateurs. Ces derniers souhaitent pouvoir héberger de nombreuses photos pour chaque événement. Les photos doivent pouvoir être ajoutées à un album ou une galerie de manière simple et intuitive. Lorsque l'on observe le système actuel, on constate que la galerie photo est composée de sous-dossiers, eux-même composés d'autres sous-dossiers. En partant de ces constatations, nous avons testé plusieurs plugins et outils tiers. Pour orienter nos recherches, nous avons utilisé une liste de plugins photo\cite{listePlugins}.\\

\underline{Lightbox Gallery}\\
Très peu de documentation en Anglais, site officiel en Japonais\\
http://wpgogo.com/development/lightbox-gallery.html

\underline{Photo Gallery}\\
\underline{Simple Photo Gallery}\\
\underline{iGalerie}\\
\underline{Next Gen Gallery}\\

=> Parler du problème de la taille des photos lors de l'upload, réglé en modifiant le php.ini


\section*{Plugin de cache}
Constat : site Web lent. Après diagnostic, on remarque que cela provient en partie de la gestion du cache.
Site de comparatif de plugins de cache : \cite{pluginsCache} \cite{pluginsCache2}

\section*{Vrac}

- Choisir un thème WP
http://www.seomix.fr/comment-choisir-theme-wordpress/


\bibliographystyle{plain}
\bibliography{biblio}

\end{document}