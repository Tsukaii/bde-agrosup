\section*{Gestion des photos}
La gestion des photos est l'un des éléments les plus délicats de ce projet. En effet, le BDE Agrosup poste un très grand nombre de photo sur son site. Il nous a donc fallu réfléchir à la meilleure solution possible. La première question a donc été de savoir où les photos allaient être hébergées. L'hébergeur actuel du site (qui sera conservé pour ce projet) proposant un espace de stockage illimité, nous en avons conlue que la meilleure solution était encore d'enregistrer les photos directement dans les dossiers du site, sans faire appel à un service d'hébergement tiers.\\

Une fois cette décision prise, nous avons réfléchi aux besoins des utilisateurs. Ces derniers souhaitent pouvoir héberger de nombreuses photos pour chaque événement. Les photos doivent pouvoir être ajoutées à un album ou une galerie de manière simple et intuitive. Lorsque l'on observe le système actuel, on constate que la galerie photo est composée de sous-dossiers, eux-même composés d'autres sous-dossiers. En partant de ces constatations, nous avons testé plusieurs plugins et outils tiers. Pour orienter nos recherches, nous avons utilisé une liste de plugins photo\cite{listePlugins}.\\

\underline{Lightbox Gallery}\\
Très peu de documentation en Anglais, site officiel en Japonais\\
http://wpgogo.com/development/lightbox-gallery.html

\underline{Photo Gallery}\\
\underline{Simple Photo Gallery}\\
\underline{iGalerie}\\
\underline{Next Gen Gallery}\\

=> Parler du problème de la taille des photos lors de l'upload, réglé en modifiant le php.ini
